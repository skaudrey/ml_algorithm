%%%%%%%%%%%%%%%%%%%%%%%%%%%%%%%%%%%%%%%%%
% University/School Laboratory Report
% LaTeX Template
% Version 3.1 (25/3/14)
%
% This template has been downloaded from:
% http://www.LaTeXTemplates.com
%
% Original author:
% Linux and Unix Users Group at Virginia Tech Wiki 
% (https://vtluug.org/wiki/Example_LaTeX_chem_lab_report)
%
% License:
% CC BY-NC-SA 3.0 (http://creativecommons.org/licenses/by-nc-sa/3.0/)
%
%%%%%%%%%%%%%%%%%%%%%%%%%%%%%%%%%%%%%%%%%

%----------------------------------------------------------------------------------------
%	PACKAGES AND DOCUMENT CONFIGURATIONS
%----------------------------------------------------------------------------------------

\documentclass{ctexart}

\usepackage[version=3]{mhchem} % Package for chemical equation typesetting
\usepackage{siunitx} % Provides the \SI{}{} and \si{} command for typesetting SI units
\usepackage{graphicx} % Required for the inclusion of images
\usepackage{natbib} % Required to change bibliography style to APA
\usepackage{amsmath} % Required for some math elements 
%\usepackage{showframe} % for showing page frames


\setlength\parindent{0pt} % Removes all indentation from paragraphs

\renewcommand{\labelenumi}{\alph{enumi}.} % Make numbering in the enumerate environment by letter rather than number (e.g. section 6)

%\usepackage{times} % Uncomment to use the Times New Roman font

%==========pesusdo codes
\usepackage[linesnumbered,boxed]{algorithm2e}
%\renewcommand{\algorithmcfname}{算法}
\renewcommand{\repeat}{Repeat}

%==========citation===========
%\usepackage{cite}
\usepackage{url}


% ==========添加首行缩进,两个字符===========
\usepackage{indentfirst}
\setlength{\parindent}{2em}
% ==========强制图片位置===========
\usepackage{float}

% ==========特殊数学符号===========
\usepackage{mathtools}
\usepackage{dsfont}
\usepackage{amsfonts}
% ==========列表===========
\usepackage{enumerate}

%----------------------------------------------------------------------------------------
%	DOCUMENT INFORMATION
%----------------------------------------------------------------------------------------

\title{采样——MCMC} % Title

%\author{fengmi \textsc{Feng}} % Author name

\author{Mia Feng} % Author name

\date{\today} % Date for the report

\begin{document}

\maketitle % Insert the title, author and date

%\begin{center}
%\begin{tabular}{l r}
%Date Performed: & January 1, 2012 \\ % Date the experiment was performed
%Partners: & James Smith \\ % Partner names
%& Mary Smith \\
%Instructor: & Professor Smith % Instructor/supervisor
%\end{tabular}
%\end{center}

% If you wish to include an abstract, uncomment the lines below
% \begin{abstract}
% Abstract text
% \end{abstract}

%----------------------------------------------------------------------------------------
%	SECTION 1
%----------------------------------------------------------------------------------------

\section{概述}
MCMC:粗暴的采样模拟方式,用于模拟直接计算困难的分布。用于采样,数值积分等等。
%To determine the atomic weight of magnesium via its reaction with oxygen and to study the stoichiometry of the reaction (as defined in \ref{definitions}):

求解目标:用多次采样得到的频率分布近似原概率分布。即本来对复杂的$f\left(x\right)$做积分,但是因为$f\left(x\right)$比较复杂所以显式积分困难。迂回方法是构造统计量$\frac{f\left(x\right)}{p\left(x\right)}$,通过对$x\sim p\left(x\right)$进行采样,求取统计量$\frac{f\left(x\right)}{p\left(x\right)}$的期望得到数值积分值。
\begin{equation}
\theta = \int_{a}^{b}f\left(x\right)\,dx=\int_{a}^{b}\frac{f\left(x\right)}{p\left(x\right)}p\left(x\right)\approx \frac{1}{n}\sum\limits_{i=0}^{n-1}\frac{f\left(x_i\right)}{p\left(x_i\right)}
\end{equation}

求解思路:微积分思想(recall:学习微积分的时候,用无数个划分的小矩形的面积来近似面积,但是当时的小矩形是来自均匀分布的)。实际上,来自均匀分布的可能性很小,此时需要考虑对复杂分布如何模拟采样,一旦完成对复杂分布的描述就可以完成数值积分。

\begin{figure}[H]
\begin{center}
\includegraphics[width=0.8\textwidth]{fig/integration.png} % Include the image placeholder.png
\caption{矩形近似}
\end{center}
\end{figure}


求解方法:Markov Chain,蒙特卡洛积分,Metroplis-Hasting,Gibbs。

思维导图:详见MCMC.xmind
\begin{figure}[H]
\begin{center}
\includegraphics[width=0.8\textwidth]{fig/Sampling.png} % Include the image placeholder.png
\caption{思维导图}
\end{center}
\end{figure}


% If you have more than one objective, uncomment the below:
%\begin{description}
%\item[First Objective] \hfill \\
%Objective 1 text
%\item[Second Objective] \hfill \\
%Objective 2 text

\subsection{基本概念}
\label{concepts}
\begin{description}
\item[马尔可夫矩阵的收敛性]
可以参考MIT的线性代数课程中对马尔可夫矩阵的讲述。马尔可夫矩阵中各元素大于0且小于1,而且矩阵是对称矩阵。马尔可夫矩阵的特征值中有一个为1,其余都是比1小的正数,所以马尔可夫矩阵的n次幂收敛至一个常数,这也是为什么马尔科夫链一定会收敛,最终可以模拟一个平稳分布的原因。

\item[马尔科夫链的细致平稳条件]
如果非周期马尔科夫链的状态转移矩阵$P$和概率分布$\pi\left(x\right)$对于所有的$i,j$满足
\begin{equation}
\pi\left(i\right)P\left(i,j\right)=\pi\left(j\right)P\left(j,i\right)
\end{equation}
则称概率分布$\pi\left(x\right)$是状态转移矩阵$P$的平稳分布。所以,利用对应的状态转移矩阵$P$,就可以模拟平稳复杂分布$\pi$。但是对于任意的平稳分布$\pi$,$P$的构造比较困难。MCMC采用迂回的方式解决了这个问题。

\item[多维数据的马尔可夫链的细致平稳条件]

平面上任意两点$E,F$,满足细致平稳条件
\begin{equation}
\pi\left(E\right)P\left(E\to F\right)=\pi\left(F\right)P\left(F\to E\right)
\end{equation}
取上一状态的条件概率分布即可作为马尔科夫链的状态转移概率。

\item[接受——拒绝采样]

当公式$\left(1\right)$中的$p\left(x\right)$不是常见分布时,无法根据$p\left(x\right)$对$x$直接进行采样。此时接受——拒绝采样可以完成对$x$的模拟采样。选定提议分布$q\left(x\right)$,如高斯分布等。先根据提议分布$q\left(x\right)$采样得到一个样本$x_0$,其对应的实际的概率是$u_0$。再从均匀分布$\left(0,kq\left(x_0\right)\right)$中采样得到一值$u$,若$u<u_0$,接受这次采样值$x_0$,反之拒绝。其中,$k,q\left(x\right)$的选取要确保$kq\left(x\right)\ge p\left(x\right)$。当然,要满足这个条件比较困难,MCMC中Metropolis-Hastings方法解决了这个问题。


\item[马尔科夫链采样]

\item[MCMC采样——MH采样]

\item[MCMC采样——Gibbs采样]



\end{description}



%----------------------------------------------------------------------------------------
%	SECTION 2
%----------------------------------------------------------------------------------------

\section{算法实现}
%见CS229\cite{stanf:cs229}
%
%\begin{enumerate}[1.]
%\item 随机初始化cluster centroids $\mu_1,\mu_2,\cdots,\mu_k\in\mathbb{R}^n$
%\item 迭代直至收敛\{
%
%对于每一个样例$i$,计算类标
%\begin{equation}
%c^{\left(i\right)}\coloneqq \arg\min\limits_{j}\big\| x^{\left(i\right)}-\mu_{j}\big\|^2
%\end{equation}
%对于每一个类$j$,更新cluster centroids:
%\begin{equation}
%\mu_j \coloneqq \frac{\sum\limits_{i=1}^{m}\mathds{1}\left\{c^{\left(i\right)}=j\right\}x^{\left(i\right)}}{\sum\limits_{i=1}^{m}\mathds{1}\left\{c^{\left(i\right)}=j\right\}}
%\end{equation}
%\}
%\end{enumerate}
注意实现时取了拉普拉斯平滑,见公式$\big(4\big)$,且为了防止下溢取对概率值取了对数。\cite{mcmc:Liu,stanf:cs229}
%----------------------------------------------------------------------------------------
%	SECTION 3
%----------------------------------------------------------------------------------------
%
\section{Implementation}
%MCMC
%\begin{figure}[H]
%\begin{center}
%\includegraphics[width=0.8\textwidth]{fig/Sampling.png} % Include the image placeholder.png
%\caption{MCMC思维导图}
%\end{center}
%\end{figure}

%%----------------------------------------------------------------------------------------
%%	SECTION 4
%%----------------------------------------------------------------------------------------
%
%\section{Results and Conclusions}
%
%The atomic weight of magnesium is concluded to be \SI{24}{\gram\per\mol}, as determined by the stoichiometry of its chemical combination with oxygen. This result is in agreement with the accepted value.
%
%\begin{figure}[h]
%\begin{center}
%\includegraphics[width=0.65\textwidth]{placeholder} % Include the image placeholder.png
%\caption{Partial Gradient of $L_\big(\theta \big)$}
%\end{center}
%\end{figure}
%
%%----------------------------------------------------------------------------------------
%%	SECTION 5
%%----------------------------------------------------------------------------------------
%
%\section{Discussion of Experimental Uncertainty}
%
%The accepted value (periodic table) is \SI{24.3}{\gram\per\mole} \cite{Smith:2012qr}. The percentage discrepancy between the accepted value and the result obtained here is 1.3\%. Because only a single measurement was made, it is not possible to calculate an estimated standard deviation.
%
%The most obvious source of experimental uncertainty is the limited precision of the balance. Other potential sources of experimental uncertainty are: the reaction might not be complete; if not enough time was allowed for total oxidation, less than complete oxidation of the magnesium might have, in part, reacted with nitrogen in the air (incorrect reaction); the magnesium oxide might have absorbed water from the air, and thus weigh ``too much." Because the result obtained is close to the accepted value it is possible that some of these experimental uncertainties have fortuitously cancelled one another.
%
%%----------------------------------------------------------------------------------------
%%	SECTION 6
%%----------------------------------------------------------------------------------------
%
%\section{Answers to Definitions}
%
%\begin{enumerate}
%\begin{item}
%The \emph{atomic weight of an element} is the relative weight of one of its atoms compared to C-12 with a weight of 12.0000000$\ldots$, hydrogen with a weight of 1.008, to oxygen with a weight of 16.00. Atomic weight is also the average weight of all the atoms of that element as they occur in nature.
%\end{item}
%\begin{item}
%The \emph{units of atomic weight} are two-fold, with an identical numerical value. They are g/mole of atoms (or just g/mol) or amu/atom.
%\end{item}
%\begin{item}
%\emph{Percentage discrepancy} between an accepted (literature) value and an experimental value is
%\begin{equation*}
%\frac{\mathrm{experimental\;result} - \mathrm{accepted\;result}}{\mathrm{accepted\;result}}
%\end{equation*}
%\end{item}
%\end{enumerate}

%----------------------------------------------------------------------------------------
%	BIBLIOGRAPHY
%----------------------------------------------------------------------------------------
%
% 注意一定要在文中引用才不会出错(至少引用一个)
\bibliographystyle{plain}
\bibliography{bib//mcmc}

%----------------------------------------------------------------------------------------


\end{document}